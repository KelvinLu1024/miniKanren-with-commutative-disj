\documentclass[format=acmlarge, review=true, authordraft=true]{acmart}

%% scheme-list :
\usepackage{listings}
\usepackage{color}
\usepackage{textcomp}

\lstset{
  language=Scheme,
  basicstyle=\ttfamily,
  morekeywords={run,conde,run*,defrel,==,fresh},
  alsodigit={!\$\%&*+-./:<=>?@^_~},
  morecomment=[l]{;,\#{}lang},
  mathescape=true
}
%% scheme-list .

% metadata

\title{miniKanren with fair search strategies}
\author{Kuang-Chen Lu}
\affiliation{Indiana University}
\author{Weixi Ma}
\affiliation{Indiana University}
\author{Daniel P. Friedman}
\affiliation{Indiana University}



%%% NOTE %%%
%
% - deadline?
%
%   miniKanren workshop deadline May 15, 2018
%
%%%%%%%%%%%%


%TODO test cases: quine, '(I love you).

%TODO 1. reformat LaTeX (80 chars per line)
%TODO 2. stat quine
%TODO 3. keep writing

% Forget the issues about efficiency
% Read the monad paper




% body
\begin{document}

\begin{abstract}

The syntax of a programming language should reflect its semantics. When using a
disjunction operator in relational programming, a programmer would expect all
clauses of this disjunction to share the same chance of being explored, as
these clauses are written in parallel. The existing multiarity disjunctive
operator in miniKanren, however, prioritize its clauses by the order of which
these clauses are written down. We have devised two new search strategies that
% allocate computational effort fairly in all clauses.
allocate computational effort more fairly in all clauses.

\end{abstract}

\maketitle

\section{introduction}

Interleaving DFS, the search strategy of \texttt{cond$^e$} in
miniKanren~\citep{friedman_reasoned_2018}, allocates computational resource
by the order in which the claueses are written. Each clause of a
\texttt{cond$^e$} takes a half of the current resource and passes the other half
to its following clauses, except for the last clause that takes all of current
resource. 
both opportunity and burden: miniKanren users can place more frequently used 
goal at the beginning to optimize their programs; however, it might be a 
catastrophe if a goal that generate many useless states is placed before more 
important goals. Seasoned miniKanreners usually know how to utilize 
the unfairness to optimize their programs. However, we believe search strategies 
that is less sensitive to goal order can also be useful to little miniKanreners 
as well as seansoned ones. We propose two such search strategies, balanced 
interleaving DFS (biDFS) and breadth-first search (BFS), and observe how they 
affect the efficiency and the answer order of known miniKanren programs. The 
experiment is conducted with the miniKanren from \textit{The Reasoned Schemer, 
2nd Edition}.

\section{related works}

% TODO rewrite, because Seres did provide a solution for possibly finite stream
% The unfairness of disjunction has been noticed by Seres et 
% al~\citet{seres1999algebra}. Their complete 
% and fair search strategy is also named breadth-first search. Their BFS 
% is fair and is similar to ours. However, their Haskell implementation cannot be 
% translated to Scheme directly. This is partially due to the difference in 
% calling convensions of host languages. Besides, their search space are infinite 
% even when no answers exists. This is not feasible in miniKanren, where 
% users can query all answers.



\section{fairness}

% KC: I changed the cost definition from ``... needed to *find* the answer''
%                                     to ``... needed to *verify* the answer''

A search strategy is \emph{fair} if answers of lower costs always come first. 
The \emph{cost} of an answer the number of relation applications needed to 
verify the answer. Now we illustrate the costs of answers by running a 
miniKanren relation. Fig.~\ref{def-repeato} defines the relation 
\texttt{repeato} that relates a term \texttt{x} with a list whose elements are 
all \texttt{x}s.

% COmmented out: This idea is borrowed from Silvija Seres's work [*].
% KC: Seres didn't phrase it this way.
%   We might need to compare our definition with Seres's one, instead of 
%   saying the idea is borrowed.

\begin{figure}
  \lstinputlisting{Figures/defrel-repeato.rkt}
  \caption{\texttt{repeato}}
  \label{def-repeato}
\end{figure}

Consider the following \texttt{run} of \texttt{repeato}.
\begin{center}
  \begin{tabular}{c}
  \lstinputlisting{Figures/run-repeato-1.rkt}
   \end{tabular}
\end{center}

The above \texttt{run} generates 4 answers. All are lists of \texttt{*}s.
The order of the answers reflects the order miniKanren discovers them:
the first answer in the list is first discovered. This order is not suprising:
to generate the first answer, \texttt{'()}, miniKanren needs to apply
\texttt{repeato} only once and the later answers need more relation
applications. In this example, the cost of each answer is the same as
one more than the number of \texttt{*}s: the cost of \texttt{'()} is 1,
the cost of \texttt{'(*)} is 2, and so on.

% A query result (list of answers) is in the \emph{cost-respecting} order if no 
% answer occurs before another answer of a lower cost. In the above example, the 
% result is cost-respecting. The iDFS search, however, does not generate 
% cost-respecting answers in general. As an example, consider the following 
% \texttt{run} of \texttt{repeato}.

In the above example, all search strategies look fair. However, the following 
example points out that iDFS is not fair.

\begin{center}
  \begin{tabular}{c}
   \lstinputlisting{Figures/run-conde-repeato-iDFS.rkt}
   \end{tabular}
\end{center}

With iDFS, \texttt{'(a a)} occurs before \texttt{'(b)} while \texttt{'(a a)} is 
associated with a higher cost. iDFS strategy is the cause, since it prioritizes 
the first \texttt{conde} case considerablely. When every \texttt{conde} case 
are equally productive, the iDFS strategy takes $1/2^{i}$ answers from the 
$i$-th case, except the last case, which share the same portion as the second 
last one. In contrast, the same run with BFS produces answers in an expected 
order.

\begin{center}
  \begin{tabular}{c}
   \lstinputlisting{Figures/run-conde-repeato-BFS.rkt}
   \end{tabular}
\end{center}


% \section{representation of search space}
% 
% ``Stream'' is often used to name the representation of search space in 
% miniKanren. However, the search space is more like a stream of list, because the 
% information that a stream is thunk is employed to redirect search effort. If the 
% search space is a normal stream, the information should have been used in a more 
% trivial way. 
% 
% In the rest of this paper, we call the \texttt{car}s of a stream 
% its \textit{mature part}, and the last \texttt{cdr} its \textit{immature part}.
% 
% \section{why disj is unfair}
% 
% The multiarity disjuction operator \texttt{disj} combine its sub-goals with 
% \texttt{disj2}, right associatively. Interestingly, although \texttt{disj} is 
% unfair, \texttt{disj2} is fair. The core functionality of \texttt{disj2} is 
% completed by \texttt{append-inf} (Fig.~ref{disj2-and-append-inf}). When both 
% input streams are infinite, the resulting mature part contains only the mature 
% part of \texttt{s-inf}. The whole \texttt{t-inf} goes to the resulting immature 
% part. However, \texttt{t-inf} and \texttt{s-inf} are swapped in the delayed 
% recursive call. Hence the search strategy spend computational effort evenly in 
% two appended streams. As \texttt{disj} applys \texttt{disj2} right 
% associatively, the left clauses are more frequently explored.
% 
% \begin{figure}
%   \lstinputlisting{Figures/disj2.rkt}
%   \caption{\texttt{disj2} and \texttt{append-inf}}
%   \label{disj2-and-append-inf}
% \end{figure}
 
\section{balanced interleaving DFS}

Our first solution, balanced interleaving DFS (biDFS), is not fair as well. 
However, it is less sensitive to goal order in disjucntion and is as efficient 
as iDFS. This search strategy is based on the  observations that a disjuction 
can be viewed as a binary tree, where \texttt{disj2}s are nodes 
and sub-goals are leaves, and that the position of a leaf has direct relation 
with its priority -- the deeper a leaf, the lower computational effort it is 
shared. In iDFS, the tree is in one of the most unbalanced forms, because 
\texttt{disj} applys \texttt{disj2} right associatively. Hence the goal order 
is very important. The reason why the tree shape can determine priority is that 
\texttt{disj2} allocates computational effort evenly to its two sub-goals.

The key idea of biDFS is to make the tree balanced (Fig.~\ref{balanced-disj}). 
We introduce a function \texttt{disj*} and its helper \texttt{split}, and change 
the \texttt{disj} macro to call \texttt{disj*} immediately. \texttt{disj*} 
essentially construct a balanced \texttt{disj2} tree. The \texttt{split} helper 
splits elements of \texttt{ls} into two lists of roughly the same length, then 
apply \texttt{k} to them.

\begin{figure}
  \lstinputlisting{Figures/balanced-disj.rkt}
  \caption{\texttt{balanced-disj}}
  \label{balanced-disj}
\end{figure}

\section{breadth-first search}

%TODO compare with Seres's approach

Breadth-first search (BFS) is fair. Our BFS is similar to the one by Seres et 
al \citet{seres1999algebra}. We conjecture that the two strategies produce
answers in the same order. ;; need more comparision

In the first subsection, we change the search strategy from iDFS to BFS. In the 
second subsection we optimize the search engine by using a queue to manage the 
search order. The two subsections corresponds to two new version of miniKanren, 
\texttt{mk-1} and \texttt{mk-2}. We refer to the original version from TRS2 as 
\texttt{mk-0}. 

% In this section we change the search strategy to \textit{breadth-first search} 
% and optimize it. The whole process is completed in two steps. In the first 
% step, from mk-0 to mk-1, BFS is introduced. In the second step, mk-1 to mk-3, 
% BFS is optimized. The initial version, \texttt{mk-0}, is exactly the version in 
% \emph{The Reasoned Schemer, 2nd Edition}.

\subsection{change search strategy from iDFS to BFS}

In both \texttt{mk-0} and \texttt{mk-1}, search spaces are represented by 
streams of  answers. The thunky streams in \texttt{mk-0} denote delayed 
computation, however, they do not necessary mean an increment in cost. We use 
the same kind of stream in \texttt{mk-1} but only put thunk at those places 
where the cost of following answers is increased by one.

To ease discussion, we call the \texttt{car}s of an stream its \emph{mature} 
part, and the last \texttt{cdr} the \emph{immature} part. When the stream is 
definitely finite, its immature part is an empty list, otherwise, it is a 
thunk. We sometimes say a stream is immature to mean its mature part is empty.

Streams denote cost correctly when they are constructed by \texttt{==}, 
\texttt{succeed}, and \texttt{fail}. However, the \texttt{mk-0} version of 
\texttt{append-inf} (Fig.~\ref{append-inf-0}) breaks cost respectiveness if its 
first input stream, \texttt{s-inf},  is infinite. The resulting mature part 
contains only the mature part of \texttt{s-inf}. If we want to encode the 
cost information correctly, the resulting mature part should also contains the 
mature part of \texttt{t-inf}.

\begin{figure}
	 \lstinputlisting{Figures/append-inf-0.rkt}	
	 \caption{\texttt{append-inf} in \texttt{mk-0}}
	 \label{append-inf-0}
\end{figure}

The \texttt{mk-1} version of \texttt{append-inf} (Fig.~\ref{append-inf-1}) 
gain fairness by combining the mature parts in the fashion of \texttt{append}. 
This \texttt{append-inf} calls its helper immediately, with the first argument, 
\texttt{s?}, set to \texttt{\#{}t}, which means \texttt{s-inf} in the helper is 
the \texttt{s-inf} in the driver. Two streams are swapped in the third 
\texttt{cond} clause, with \texttt{s?} flipped accordingly.

\begin{figure}
	 \lstinputlisting{Figures/append-inf-1.rkt}	
	 \caption{\texttt{append-inf} in \texttt{mk-1}}
	 \label{append-inf-1}
\end{figure}

\texttt{mk-1} is not efficient in two aspects. \texttt{append-inf} need to copy 
all \texttt{cons} cells of \emph{both} input streams when the first stream is 
possibly infinite. Besides, \texttt{mk-1} computes answers of the same cost
at once, even when only a portion is queried. We solves the two problems in the 
next subsections.

\subsection{optimize breadth-first search}

We avoid generating same-cost answers at once by expressing BFS with a queue, 
whose elements thunks that return a new stream. Every \texttt{mk-1} stream has 
zero or one thunk, so we cannot manage it with queue in interesting 
way. Therefore we change the representation of immature parts from thunks to 
lists of thunks. As as consequence, we also change the way to combine mature and 
immature part from \texttt{append} to \texttt{cons}. 

After applying this two changes, stream representation becomes more 
complicated. It motivates us to set up an interface between stream 
and the rest of miniKanren. Listed in Fig.~\ref{interface} are all functions 
being aware of the stream representation, but \texttt{take-inf} and its helper 
function, which are explained later. The first three functions are 
constructors: 
\texttt{empty-inf} constructs an empty stream; \texttt{unit-mature-inf} 
constructs a stream with one mature solution; \texttt{unit-immature-inf} 
constructs a stream with one thunk. The \texttt{append-inf} in \texttt{mk-3} is 
relatively straightforwared compared with the \texttt{mk-1} version. 
\texttt{append-map-inf} is more tricky on how to construct the new immature 
part. We can follow the approach in mk-0 and mk-1 -- create a new thunk which 
invoke append-map-inf recursively when forced. But then we need to be careful: 
if we construct the thunk when the old immature part is an empty list, the 
resulting stream might be infinitely unproductive. Beside, all solutions of the 
next lowest cost in \texttt{s-inf} must be computed when the thunk is invoked. 
However sometimes only a portion of these solutions is required to answer a 
query. To avoid the trouble and the too-early computation, we choose to create 
a new thunk for every existing thunk. The next four functions are used only by 
\texttt{ifte} and \texttt{once}. Uninterested readers might skip them. 
\texttt{null-inf?} checks whether a stream is exausted. \texttt{mature-inf?} 
checks whether a stream has some mature solutions. \texttt{car-inf} takes the 
first solution out of a mature stream. \texttt{cdr-inf} drops the first 
solution of a mature stream. Finally, \texttt{force-inf} forces an a immature 
stream to do more computation.

\begin{figure}
	 \lstinputlisting{Figures/interface.rkt}	
	 \caption{Functions being aware of stream representation}
	 \label{interface}
\end{figure}

The last interesting function is \texttt{take-inf} (Fig.~\ref{take-inf-3-1}). 
The parameter \texttt{vs} is a list of solutions. The next two parameters, 
\texttt{P} and \texttt{Q}, together represent a queue. The first two 
\texttt{cond} lines are very similar to their counterparts in mk-0 and mk-1. 
The third line runs when we exaust all solutions. The forth line re-shape the 
queue. The fifth and last line invoke the first thunk in the queue and use the 
mature part of the resulting stream, \texttt{s-inf}, as the new \texttt{vs}, 
and enqueuing \texttt{s-inf}'s thunks. 

\begin{figure}
	 \lstinputlisting{Figures/take-inf-3-1.rkt}	
	 \caption{\texttt{take-inf} in \texttt{mk-3-1}}
	 \label{take-inf-3-1}
\end{figure}



\section{conclusion}

\section*{acknowledgments}

\bibliographystyle{ACM-Reference-Format}
\bibliography{citation}

\end{document}

