\documentclass[format=acmlarge, review=true, authordraft=true]{acmart}

%% scheme-list :
\usepackage{listings}
\usepackage{color}
\usepackage{textcomp}
\usepackage{trackchanges}

\addeditor{MVC}
\addeditor{LKC}

\lstset{
  language=Scheme,
  basicstyle=\ttfamily,
  morekeywords={run,conde,run*,defrel,==,fresh},
  alsodigit={!\$\%&*+-./:<=>?@^_~},
  morecomment=[l]{;,\#{}lang},
  mathescape=true
}
%% scheme-list .

% metadata

\title{miniKanren with fair search strategies}
\author{Kuang-Chen Lu}
\affiliation{Indiana University}
\author{Weixi Ma}
\affiliation{Indiana University}
\author{Daniel P. Friedman}
\affiliation{Indiana University}




%%% NOTE %%%
%
%  SUBMISSION DEADLINE: May 15, 2019
%
%%%%%%%%%%%%


\newcommand{\conde}{\texttt{cond$^e$} }
\newcommand{\conj}{\texttt{conj}}
\newcommand{\disj}{\texttt{disj}}


\newtheorem{defn}{Definition}[section]

\begin{document}

\begin{abstract}

The syntax of a programming language should reflect its semantics. When writing 
a \conde expression in miniKanren, a programmer would expect all clauses share 
the same chance of being explored, as these clauses are written in 
parallel. The existing search strategy, interleaving DFS (DFS$_\textrm{i}$), 
however, prioritize its clauses by the order how they are written down. 
Similarly, when a \conde is followed by another goal conjunctively, a 
programmer would expect answers in parallel share the same chance of being 
explored. Again, the answers by DFS$_\textrm{i}$ is different from the 
expectation. We have devised three new search strategies that have different 
level of fairness in \disj and \conj.

%The syntax of a programming language should reflect its semantics. When using 
%a disjunctive operator in relational programming, a programmer would expect 
%all 
%sub-goals of this disjunct to share the same chance of being explored, as 
%these 
%clauses are written in parallel. The existing multi-arity disjunctive operator 
%in miniKanren, however, prioritize its clauses by the order of which these 
%clauses are written down. We have devised two new search strategies that 
%allocate computational effort more fairly in all clauses.

\end{abstract}

\maketitle

\section{introduction}

miniKanren is a family of relational programming languages. miniKanren 
programs, especially relational interpreters, have been proven to be 
useful in solving many problems by \citet{byrd2017unified}. 

A subtlety in writing miniKanren programs with large \conde expressions, such 
as relational interpreters, is that the order of \conde clauses sometimes 
affect the speed considerably. This phenomenon appears when running with the 
miniKanren in \citet{friedman_reasoned_2018}, one of the most well-understood 
implementation. This is because for all \conde expressions with more than two 
clauses, the clauses are given different ``search priority''. Left clauses are 
given higher priorities. This biased treatment causes 2 problems when a \conde 
expression has many clauses: the right-most clause can hardly contribute to 
query result; and the efficiency of the whole program might depend largely on 
the order of these clauses.

\NOTE[LKC]{Should I explain ``state'' and ``search space'' here?}

Under the hood, \conde clauses are combined with \disj{}. The \disj{} 
implementation in \citep{friedman_reasoned_2018} is not \emph{fair}.

\begin{defn}[fair \disj]
	A \disj{} is fair iff it allocates computational resource evenly among 
	search spaces derived from disjunctive goals. \citet{seres1999algebra}
\end{defn}

A closely related concept is \emph{almost-fair} \disj.

\begin{defn}[almost-fair \disj]
	A \disj{} is almost-fair iff it allocates computational resource so evenly 
	among search spaces derived from disjunctive goals that the maximal ratio 
	of resources is bounded by a constant.
\end{defn}

The \disj{} in \citet{friedman_reasoned_2018}, or more precisely, in its search 
strategy, interleaving DFS(DFS$_\textrm{i}$), is neither fair nor almost-fair. 
Fair depth-first search (DFS$_\textrm{f}$), a new search strategy in this 
paper, has fair \disj. And balanced interleaving depth-first search 
(DFS$_\textrm{bi}$), another new strategy, has almost-fair \disj. Breath-first 
search (BFS), a known strategy \citet{seres1999algebra}, has fair \disj{} and 
\emph{fair \conj{}}.

\begin{defn}[fair \conj]
	A \conj{} is fair iff it allocates computational resource evenly among 
	search spaces derived from states in the same bag, where bags are finite 
	list of states.
\end{defn}

A comparison of fairness of search strategies is in Fig.~\ref{fairness}. 

To summarize our contribution, we
\begin{itemize}
	\item propose a new concept, almost-fair \disj{}.
	\item propose a new definition of fair \conj{}.
	\item propose and implement balanced interleaving depth-first search 
	(DFS$_\textrm{bi}$), a new search strategy with almost-fair \disj{}.
	\item propose and implement fair depth-first search (DFS$_\textrm{f}$), a 
	new search strategy with fair \disj{}.
	\item implement in a new way breath-first search (BFS), a search strategy 
	with fair \disj{} and fair \conj{}
	(our code runs faster in all benchmarks and is simpler)
	\item prove our BFS implementation is equivalent with the one by 
	\citet{seres1999algebra}.
\end{itemize}
 
\begin{figure}[tbp]
	\begin{tabular}{|c|c|c|c|c|}
		\hline 
		fairness & DFS$_\textrm{i}$ & DFS$_\textrm{bi}$ & DFS$_\textrm{f}$ & BFS \\ 
		\hline 
		disj & unfair & almost-fair & fair & fair \\ 
		\hline 
		conj & unfair & unfair & unfair & fair \\ 
		\hline 
	\end{tabular} 
	\caption{fairness of search strategies}
	\label{fairness}
\end{figure}

% Interleaving DFS, the search strategy of \conde in
% miniKanren~\citet{friedman_reasoned_2018}, allocates computational resource
% by the order in which the clauses are written. Each clause of a
% \conde takes half of the current resource and passes the other half
% to its following clauses, except for the last clause that takes all of the 
%current resource. 
% The biased treatment provides both opportunity and burden: miniKanren users 
%can place more frequently used 
% goal at the beginning to optimize their programs; however, it might be a 
% catastrophe if a goal that generates many useless states is placed before 
%more 
% important goals. Seasoned miniKanreners usually know how to utilize 
% the unfairness to optimize their programs. However, we believe search 
%strategies that is less sensitive to goal order can also be useful to little 
%miniKanreners as well as seasoned ones. We propose two such search strategies, 
%balanced interleaving DFS (DFS$_\textrm{bi}$) and breadth-first search (BFS), and observe 
%how they affect the efficiency and the answer order of known miniKanren 
%programs. The experiment is conducted with the miniKanren from \textit{The 
%Reasoned Schemer, 2nd Edition}.

\section{fairness}

In this section, we elaborate fairness by running queries about 
\texttt{repeato}, a relational definition that relates a term \texttt{x} with a 
non-empty list whose elements are \texttt{x} (Fig.~\ref{repeato}).

\begin{figure}
	\lstinputlisting{Figures/repeato.rkt}
	\caption{\texttt{repeato} and an example run}
	\label{repeato}
\end{figure}

\subsection{fair \texttt{disj}}

Given the following program, it is natural to expect lists of each letter to
constitute $1/4$ in the answer. DFS$_\textrm{i}$, the current search
strategy, however, results in many more lists of \texttt{a}s than lists
of other letters. And some letters, e.g. \texttt{c} and \texttt{d}, are
rarely seen. The situation would be exacerbated if \conde contains more clauses.

\begin{center}
	\begin{tabular}{c}
		\lstinputlisting{Figures/repeato-disj-iDFS.rkt}
	\end{tabular}
\end{center}

We borrow the definition of \emph{fair} \disj{} from \citet{seres1999algebra}: 
search strategies with \emph{fair} \disj{} should allocate resources evenly 
among disjunctive goals. Running the same program with DFS$_\textrm{f}$ and BFS 
give the following result. 

\begin{center}
	\begin{tabular}{c}
		\lstinputlisting{Figures/repeato-disj-fair.rkt}
	\end{tabular}
\end{center}

Now we are in a middle place between fair and unfair -- search strategies
with  \emph{almost-fair} \disj{} should allocate resources so evenly among
disjunctive goals that the maximal ratio of resources is bounded by a constant. 
Our new search strategy,
DFS$_\textrm{bi}$, has almost-fair \disj{}. It is
fair when the number of goals is a power of 2, otherwise, some goals are 
allocated twice as many resources than the others. In the previous example,
DFS$_\textrm{bi}$
gives the same result. And in the following example, where the \conde has 5 
clause, the clauses of \texttt{b}, \texttt{c}, and \texttt{d} are allocated 
more resources.

\begin{center}
	\begin{tabular}{c}
		\lstinputlisting{Figures/repeato-disj-biDFS.rkt}
	\end{tabular}
\end{center}

\subsection{fair \texttt{conj}}

In the following program, the three \conde clauses differ in a trivial way. So 
we expect lists of each letter constitute $1/4$ of the answer list. Search 
strategies with unfair \conj{} (e.g. DFS$_\textrm{i}$, DFS$_\textrm{f}$), 
however, give us many more lists of \texttt{a}s than lists of other letters. 
And some letters (e.g. lists 
of \texttt{c}) are rarely found. Although DFS$_\textrm{i}$'s \disj{} is unfair in general, 
it is fair when there is no call to relational definition in sub-goals, 
including this case. The situation would be worse if we add more \conde 
clauses. The result with DFS$_\textrm{bi}$, whose \conj{} is also unfair, is similar, but 
due to its different \disj{}, the position of \texttt{b} and \texttt{c} are 
swapped. 

\begin{center}
	\begin{tabular}{c}
		\lstinputlisting{Figures/repeato-conj-unfair.rkt}
	\end{tabular}
\end{center}

Intuitively, search strategies with fair \conj{} should produce each letter of 
lists equally frequently. Indeed, BFS does so.

\begin{center}
	\begin{tabular}{c|c}
		\lstinputlisting{Figures/repeato-conj-fair.rkt}
	\end{tabular}
\end{center}

A more interesting situation is when the first conjunctive goal produces 
infinite many answers. Consider the following example, a naive specification of 
fair \conj{} might require search strategies to produce all sorts of singleton 
lists, but no longer ones, which makes the strategies \emph{incomplete}. 
\NOTE[MVC]{incomplete w.r.t. what? where's the definition of incomplete?}
\NOTE[LKC]{I am a bit confused. I assume completeness is a well-known concept 
in the context of logic programming. For example, this paper doesn't cite any 
source when it talks about completeness.

Hemann, Jason, et al. "A small embedding of logic programming with a simple 
complete search." ACM SIGPLAN Notices. Vol. 52. No. 2. ACM, 2016.}

\begin{center}
	\begin{tabular}{c}
		\lstinputlisting{Figures/repeato-conj-infinite-naive.rkt}
	\end{tabular}
\end{center}

Our solution requires a search strategy with \emph{fair} \conj{} to 
package answers in bags, where each bag contains finite answers, and to allocate 
resources evenly among search spaces derived from answers in the same bag. The 
way to package depends on search strategy. And how to allocate resources among 
search space related to different bags is unspecified. Our definition of fair 
\conj{} is orthogonal with completeness. For example, a naively fair strategy 
is fair but not complete, while BFS is fair and complete. 
\NOTE[MVC]{Also here, complete w.r.t what?}

BFS packages answers by their costs. The \emph{cost} of a answer is its 
depth in the search tree (i.e. the number of calls to relational definitions 
required to find them) \citet{seres1999algebra}. In the following example, 
every answer is a list of list of symbol. The cost of each of them is equal to 
the length of the inner lists plus the length of the outer list. In addition to 
being fair, BFS also produces answers in increasing order of cost.
\NOTE[MVC]{Here inner and outer are very confusing. Can you be more specified?}
\NOTE[LKC]{updated}

\begin{center}
	\begin{tabular}{c}
		\lstinputlisting{Figures/repeato-conj-infinite-sof.rkt}
	\end{tabular}
\end{center}

\section{balanced interleaving depth-first search}

\begin{figure}
	\lstinputlisting{Figures/balanced-disj.rkt}
	\caption{DFS$_\textrm{bi}$ implementation}
	\label{balanced-disj}
\end{figure}

Balanced interleaving DFS (DFS$_\textrm{bi}$) has almost-fair \disj{} and unfair \conj{}. 
The implementation of DFS$_\textrm{bi}$ differs from DFS$_\textrm{i}$ in the \disj{} macro. We list the 
new \disj{} with its helpers in Fig.~\ref{balanced-disj}. The first helper 
function, \texttt{split}, takes a list of goals \texttt{ls} and a procedure 
\texttt{k}, partitions \texttt{ls} into two sub-lists of roughly equal length, 
and returns the application of \texttt{k} to the two sub-lists. \texttt{disj*} 
takes a non-empty list of goals \texttt{gs} and returns a goal. With the help 
of \texttt{split}, it essentially constructs a \emph{balanced} binary tree 
where leaves are elements of \texttt{gs} and nodes are \texttt{disj2}s, hence 
the name of this search strategy. In contrast, the \disj{} in DFS$_\textrm{i}$
constructs the binary tree with the same nodes but in the unbalanced form.

\section{fair depth-first search}

\begin{figure}
	\lstinputlisting{Figures/fDFS.rkt}
	\caption{DFS$_\textrm{f}$ implementation}
	\label{fDFS}
\end{figure}

Fair DFS (DFS$_\textrm{f}$) has fair \disj{} and unfair \conj{}. The implementation of DFS$_\textrm{f}$ 
differs from DFS$_\textrm{i}$'s in \texttt{disj2} (Fig.~\ref{fDFS}). \texttt{disj2} is 
changed to call a new and fair version of \texttt{append-inf}. 
\texttt{append-inf/fair} immediately calls its helper, 
\texttt{append-inf/fair\^{}}, with the first argument, \texttt{s?}, set to 
\texttt{\#{}t}, which indicates that \texttt{s-inf} and \texttt{t-inf} haven't 
been swapped. The swapping happens at the third \texttt{cond} clause in the 
helper, where \texttt{s?} is updated accordingly. The first two \texttt{cond} 
clauses essentially copy the \texttt{car}s and stop recursion when one of the 
input spaces is obviously finite. The third clause, as we mentioned above, is 
just for swapping. When the fourth and last clause runs, we know that both 
\texttt{s-inf} and \texttt{t-inf} are ended with a thunk. In this case, a new 
thunk is constructed. The new thunk calls the driver recursively. Here changing 
the order of \texttt{t-inf} and \texttt{s-inf} won't hurt the fairness (though 
it will change the order of answers). We swapped them back so that answers are 
produced in a more natural order.


\section{breadth-first search}

BFS is fair in both \disj{} and \conj{}. Our implementation is based on 
DFS$_\textrm{f}$ (not DFS$_\textrm{i}$). All we have to do is apply two trivial changes to 
\texttt{append-map-inf}. First, rename it to \texttt{append-map-inf/fair}. 
Second, replace its use of \texttt{append-inf} to \texttt{append-inf/fair}. 

The implementation can be improved in two ways. First, as mentioned in 
section 2.2, BFS puts answers in bags and answers of the same cost are in the 
same bag. In this implementation, however, it is unclear where this information 
is recorded. Second, 
\texttt{append-inf/fair} is extravagant in memory usage. It makes 
$O(n+m)$ new \texttt{cons} cells every time, where $n$ and $m$ are the 
``length''s of input search spaces. We address these issues in the first 
subsection.

Both our BFS and Seres's BFS \citet{seres1999algebra} produce answers in 
increasing order of cost. So it is interesting to see if they are equivalent. 
We prove so in Coq. The details are in the second subsection.

\subsection{optimized BFS}

\begin{figure}
	\lstinputlisting{Figures/BFS-opt.rkt}	
	\caption{new and changed functions in optimized BFS that implements pure 
	features}
	\label{BFS-opt}
\end{figure}

\NOTE[MVC]{Though bag is well known, people rarely say ``bagging''. How about putting information in a bag, or something better?}
\NOTE[MVC]{What is the bagging information?}
\NOTE[LKC]{It's just cost... You're right. I should have be more direct.}

As mentioned in section 2.2, BFS puts answers in bags and answers of the 
same cost are in the same bag. The cost
information is recorded subtly -- the \texttt{car}s of a search space have cost 
0 (i.e. they are in the same bag), and the costs of answers in thunk are 
computed recursively then increased by one. It is even more subtle that
\texttt{append-inf/fair} and the \texttt{append-map-inf/fair} respects the cost 
information. We make these facts more obvious by changing the type of search 
space, modifying related function definitions, and introducing a few more 
functions.

The new type is a pair whose \texttt{car} is a list of answers (the bag), and 
whose \texttt{cdr} is either a \texttt{\#{}f} or a thunk returning a search 
space. A falsy \texttt{cdr} means the search space is obviously finite. 

Functions related to the pure subset are listed in Fig.~\ref{BFS-opt} (the 
others in Fig.~\ref{BFS-opt-cont}). They are compared with 
\citeauthor{seres1999algebra}'s implementation later. The first three functions 
in Fig.~\ref{BFS-opt} are search space constructors. \texttt{none} makes an 
empty search space; \texttt{unit} makes a space from one answer; and 
\texttt{step} makes a space from a thunk. The remaining functions do the same 
thing as before. 

Luckily, the change in \texttt{append-inf/fair} also fixes the miserable space 
extravagance -- the use of \texttt{append} helps us to reuse the first bag of 
\texttt{t-inf}.

\citet{kiselyov2005backtracking} has shown that a \emph{MonadPlus} hides in 
implementations of logic programming system. Our BFS implementation is not an 
exception: \texttt{none}, \texttt{unit}, \texttt{append-map-inf}, and 
\texttt{append-inf} correspond to \texttt{mzero}, \texttt{unit}, \texttt{bind}, 
and \texttt{mplus} respectively.

\begin{figure}
	\lstinputlisting{Figures/BFS-opt-cont.rkt}	
	\caption{new and changed functions in optimized BFS that implements impure 
		features}
	\label{BFS-opt-cont}
\end{figure}

Functions implementing impure features are in Fig.~\ref{BFS-opt-cont}. The 
first function, \texttt{elim}, takes a space \texttt{s-inf} and two 
continuations \texttt{ks} and \texttt{kf}. When \texttt{s-inf} contains some 
answers, \texttt{ks} is called with the first answer and the rest space. 
Otherwise, \texttt{kf} is called with no argument. Here `s' and `f' means 
`succeed' and `fail' respectively. This function is an eliminator of search 
space, hence the name. The remaining functions do the same thing as before.

\subsection{comparison with the BFS of \citet{seres1999algebra}}

In this section, we compare the pure subset of our optimized BFS with the BFS 
found in \citet{seres1999algebra}. We focus on the pure subset because 
Silvija's system is pure. Their system represents search spaces with streams of 
lists of answers, where each list is a bag.

To compare efficiency, we translate her Haskell code into Racket (See 
supplements for the translated code). The translation is direct 
due to the similarity in both logic programming systems and search space 
representations. The translated code is longer and slower. Details about 
difference in efficiency are in section 6.

We prove in Coq that the two BFSs are equivalent, i.e. \texttt{(run n g)} 
produces the same result (See supplements for the formal proof).

\section{quantitative evaluation}

\begin{table}
	\begin{tabular}{|c|c|c|c|c|c|c|}
		\hline 
		benchmark & size & DFS$_\textrm{i}$ & DFS$_\textrm{bi}$ & DFS$_\textrm{f}$ & optimized BFS & Silvija's BFS  
		\\
		\hline
		very-recursiveo & 100000 &  579 &  793 & 2131 & 1438 & 3617 \\
		& 200000 & 1283 & 1610 & 3602 & 2803 & 4212 \\
		& 300000 & 2160 & 2836 &    - & 6137 &    - \\
		\hline 
		appendo  & 100 &  31 &  41 &  42 &  31 &  68 \\ 
		& 200 & 224 & 222 & 221 & 226 & 218 \\ 
		& 300 & 617 & 634 & 593 & 631 & 622 \\ 
		\hline 
		reverso & 10 &   5 &   3 &   3 &     38 &     85 \\ 
		& 20 & 107 &  98 &  51 &   4862 &   5844 \\
		& 30 & 446 & 442 & 485 & 123288 & 132159 \\ 
		\hline
		quine-1 & 1 &  71 &  44 & 69 & - & - \\ 
		& 2 & 127 & 142 & 95 & - & - \\ 
		& 3 & 114 & 114 & 93 & - & - \\ 
		\hline
		quine-2 & 1 & 147 & 112 &  56 & - & - \\ 
		& 2 & 161 & 123 & 101 & - & - \\ 
		& 3 & 289 & 189 & 104 & - & - \\ 
		\hline 
		'(I love you)-1 &  99 & 56 & 15 & 22 &  74 & 165 \\ 
		& 198 & 53 & 72 & 55 &  47 &  74 \\
		& 297 & 72 & 90 & 44 & 181 & 365 \\ 
		\hline
		'(I love you)-2 &  99 & 242 &  61 & 16 &  66 &  99 \\ 
		& 198 & 445 & 110 & 60 &  42 &  64 \\
		& 297 & 476 & 146 & 49 & 186 & 322 \\ 
		\hline 
	\end{tabular}
	\caption{The results of a quantitative evaluation: running times of 
	benchmarks 
		in milliseconds}
	\label{compare-efficiency}
\end{table}

In this section, we compare the efficiency of search strategies. A concise 
description is in Table~\ref{compare-efficiency}. A hyphen means running out of 
memory. The first two benchmarks are taken from 
\citet{friedman_reasoned_2018}. \texttt{reverso} is from 
\citet{rozplokhas2018improving}. Next two benchmarks 
about quine are modified from a similar test case in \citet{byrd2017unified}. 
The modifications are made 
to circumvent the need for symbolic constraints (e.g. $\neq$, 
\texttt{absento}). Our version generates de 
Bruijnized expressions and prevent closures getting into list. The two 
benchmarks differ in the \conde clause order of their relational interpreters. 
The last two 
benchmarks are about synthesizing expressions that evaluate to \texttt{'(I love 
you)}. This benchmark is also inspired by \citet{byrd2017unified}. Again, the 
sibling benchmarks differ in the \conde clause order of their relational 
interpreters. The first one 
has elimination rules (i.e. application, \texttt{car}, and \texttt{cdr}) at the 
end, while the other has them at the beginning. We conjecture that DFS$_\textrm{i}$ would 
perform badly in the second case because elimination rules complicate the 
problem when running backward. The evaluation supports our conjecture.

In general, only DFS$_\textrm{i}$ and DFS$_\textrm{bi}$ constantly perform well. DFS$_\textrm{f}$ is just as 
efficient in all benchmarks but \texttt{very-recursiveo}. Both BFS have obvious 
overhead in many cases. Among the three variants of DFS (they all have unfair 
\conj{}), DFS$_\textrm{f}$ is most resistant to clause permutation, followd by DFS$_\textrm{bi}$ then 
DFS$_\textrm{i}$. Among the two implementation of BFS, ours constantly performs as well or 
better. Interestingly, every strategies with fair \disj{} suffers in 
\texttt{very-recursiveo} and DFS$_\textrm{f}$ performs well elsewhere. 
Therefore, this 
benchmark might be a special case. Fair \conj{} imposes overhead constantly 
except in \texttt{appendo}. The reason might be that strategies with fair 
\conj{} tend to keep more intermediate answers in the memory.

\section{related works}

Edward points out a disjunct complex would be `fair' if it is a full and 
balanced tree \citet{yang2010adventures}.

Silvija et al \citet{seres1999algebra} also describe a breadth-first search 
strategy. We proof their BFS is equivalent to ours. But our code looks simpler 
and performs better in comparison with a straightforward translation of their 
Haskell code.

\section{conclusion}

We analysis the definitions of fair \disj{} and fair \conj{}, then propose a 
new definition of fair \conj{}. Our definition is orthogonal with completeness.

We devise three new search strategies: balanced interleaving DFS (DFS$_\textrm{bi}$), fair 
DFS (DFS$_\textrm{f}$), and BFS. DFS$_\textrm{bi}$ has almost-fair \disj{} and unfair \conj{}. DFS$_\textrm{f}$ has 
fair \disj{} and unfair \conj{}. BFS has both fair \disj{} and fair \conj{}.

Our quantitative evaluation shows that DFS$_\textrm{bi}$ and DFS$_\textrm{f}$ are competitive 
alternatives to DFS$_\textrm{i}$, the current search strategy, and that BFS is less 
practical.

We prove our BFS is equivalent to the BFS in \citet{seres1999algebra}. Our code 
is shorter and runs faster than a direct translation of their Haskell code.

\section*{acknowledgments}

\bibliographystyle{ACM-Reference-Format}
\bibliography{citation}

\end{document}

