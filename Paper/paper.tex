\documentclass[format=acmlarge, review=true, authordraft=true]{acmart}

%% scheme-list :
\usepackage{listings}
\usepackage{color}
\usepackage{textcomp}
\usepackage{trackchanges}

\addeditor{MVC}
\addeditor{LKC}

\lstset{
  language=Scheme,
  basicstyle=\ttfamily,
  morekeywords={run,conde,run*,defrel,==,fresh},
  alsodigit={!\$\%&*+-./:<=>?@^_~},
  morecomment=[l]{;,\#{}lang},
  mathescape=true
}
%% scheme-list .

% metadata

\title{miniKanren with fair search strategies}
\author{Kuang-Chen Lu}
\affiliation{Indiana University}
\author{Weixi Ma}
\affiliation{Indiana University}
\author{Daniel P. Friedman}
\affiliation{Indiana University}




%%% NOTE %%%
%
%  SUBMISSION DEADLINE: May 15, 2019
%
%%%%%%%%%%%%


\newcommand{\conde}{\texttt{cond$^e$}}
\newcommand{\conj}{\texttt{conj}}
\newcommand{\disj}{\texttt{disj}}
\newcommand{\clisting}[1]{
\begin{center}
  \begin{tabular}{c}
	\lstinputlisting{#1}
    \end{tabular}
\end{center}
}
\newcommand{\DFSi }[0]{DFS$_\textrm{i}$}
\newcommand{\DFSf }[0]{DFS$_\textrm{f}$}
\newcommand{\DFSbi}[0]{DFS$_\textrm{bi}$}
\newcommand{\BFS  }[0]{BFS}


\newtheorem{defn}{Definition}[section]

\begin{document}

\begin{abstract}

The syntax of a programming language should reflect its semantics. When writing 
a \conde{} expression in miniKanren, a programmer would expect all clauses 
share the same chance of being explored, as these clauses are written in 
parallel. The existing search strategy, interleaving depth-first search 
(\DFSi{}), 
however, prioritize its clauses by the order how they are written down. 
Similarly, when a \conde{} is followed by another goal conjunctively, a 
programmer would expect answers in parallel share the same chance of being 
explored. Again, the answers by \DFSi{} is different from the 
expectation. We have devised three new search strategies that have different 
level of fairness in \disj{} and \conj{}.

\NOTE[LKC]{\disj{} and \conj{} occurs free in the abstract.}

%The syntax of a programming language should reflect its semantics. When using 
%a disjunctive operator in relational programming, a programmer would expect 
%all 
%sub-goals of this disjunct to share the same chance of being explored, as 
%these 
%clauses are written in parallel. The existing multi-arity disjunctive operator 
%in miniKanren, however, prioritize its clauses by the order of which these 
%clauses are written down. We have devised two new search strategies that 
%allocate computational effort more fairly in all clauses.

\end{abstract}

\maketitle

\section{introduction}

miniKanren is a family of relational programming languages.
\citet{friedman_reasoned_2018} introduce miniKanren and its implementation
in \emph{The Reasoned Schemer, 2nd Ed} (TR2). miniKanren programs, especially
relational interpreters, have been proven to be useful in solving many problems 
by \citet{byrd2017unified}. 

% \conde is the flagship disjunction operator in miniKanren.
A subtlety arises when a \conde contains many clauses: not every clause has an 
equal chance to contribute to the result. As an example, consider the following 
relation \texttt{repeato} and its invocation. 

\clisting{Figures/repeato.rkt}

Next, consider the following disjunction of invoking \texttt{repeato} with four 
different letters.

\clisting{Figures/example.rkt}

\conde intuitively relates its clauses with logical \texttt{or}. And thus an 
unsuspicious beginner would expect each letter to contribute equally to the 
result, as follows.

\clisting{Figures/run-repeato-fair.rkt}

The \conde in TR2, however, generates a less expected result.

\clisting{Figures/run-repeato-idfs.rkt}

The miniKanren in TR2 implements interleaving DFS (\DFSi), the cause of this 
unexpected result. With this search strategy, each clause takes half 
of its received computational resources and pass the other half to its 
following clauses, except for the last clause that takes all resources it 
receives. In the example above, the \texttt{a} clause takes half of all 
recourses. And the \texttt{b} clause takes a quarter. Thus \texttt{c} and 
\texttt{d} barely contribute to the result.

%In the example above, the letters \texttt{c} and \texttt{d} are 
%allocated with less resources than \texttt{a}, and thus \texttt{c} and 
%\texttt{d} barely contribute to the result.

\DFSi{} is sometimes powerful for an expert. By carefully organizing the order 
of \conde{} clauses, a miniKanren program can explore more ``interesting'' 
clauses than those uninteresting ones, and thus use computational resources 
efficiently. A little miniKanrener, however, may beg to differ--understanding
implementation details and fiddling with clauses order is not the first
priority of a beginner.

There is another reason that miniKanren could use more search strategies than
just \DFSi. In many applications, there does not exist one order that serves
for all purposes. For example, a relational dependent type checker contains
clauses for constructors that build data and clauses for eliminators that use
data. When the type checker is used to generate simple and shallow programs,
the clauses of constructors should be put in the front of \conde.
When performing proof searches for complicated programs, the clauses of 
eliminator should take the focus. With \DFSi, these two uses cannot be 
efficient at the same time. In fact, to make one use efficient, the other one 
must be drastically slow.

The specification that every clause in the same \conde{} is given equal 
``search priority'' is called fair \disj{}. And search strategies with 
almost-fair \disj{} give every clause in the same \conde{} similar priority. 
Fair \conj{}, a related concept, is more complicated. We defer it to the next 
section.

To summarize our contribution, we
\begin{itemize}
	\item propose and implement balanced interleaving depth-first search 
	(\DFSbi{}), a new search strategy with almost-fair \disj{}.
	\item propose and implement fair depth-first search (\DFSf{}), a 
	new search strategy with fair \disj{}.
	\item implement in a new way breath-first search (BFS), a search strategy 
	with fair \disj{} and fair \conj{}. (our code runs faster in all 
benchmarks and is simpler). And we prove formally that our BFS implementation 
is equivalent to the one by \citet{seres1999algebra}.
\end{itemize}

\section{search strategies and fairness}

In this section, we define fair \disj{}, almost-fair \disj{} and fair \conj{}. 
Before going further into fairness, we would like to give a short review about 
state, search space, and goal, because fairness is defined in terms of them. 
A \emph{state} is a collection of constraints. Every answer corresponds to a 
state. A \emph{search space} is a collection of states. And a \emph{goal} is a 
function from a state to a search space. 

% Goals check their input states and outputs possibly extended states. A goal 
% might fail for some inputs, in which case the output search space would be 
% empty.

Now we elaborate fairness by running more queries about \texttt{repeato}.

\subsection{fair \texttt{disj}}

Given the following program, it is natural to expect lists of each letter to
constitute $1/4$ in the answer. \DFSi, the current search
strategy, however, results in many more lists of \texttt{a}s than lists
of other letters. And some letters  (e.g. \texttt{c} and \texttt{d}) are
rarely seen. The situation would be exacerbated if \conde contains more clauses.

\begin{center}
	\begin{tabular}{c}
		\lstinputlisting{Figures/repeato-disj-DFSi.rkt}
	\end{tabular}
\end{center}

Under the hood, the \conde here is allocating computational resource to 
four trivially different search space. The unfair \disj{} in 
\DFSi{} allocates many more resources to the first search space. On the 
contrary, fair \disj{} would allocate resources evenly to each search space. 

\begin{center}
	\begin{tabular}{c}
		\lstinputlisting{Figures/repeato-disj-DFSf.rkt}
	\end{tabular}
\end{center}
\begin{center}
\begin{tabular}{c}
	\lstinputlisting{Figures/repeato-disj-BFS.rkt}
\end{tabular}
\end{center}

Running the same program again with almost-fair \disj {} (e.g. 
DFS$_\textrm{bi}$) gives the same result. Almost-fair, however, is not 
completely fair, as shown by the following example. 

\begin{center}
	\begin{tabular}{c}
		\lstinputlisting{Figures/repeato-disj-DFSbi.rkt}
	\end{tabular}
\end{center}

\DFSbi{} is fair only when the number of goals is a power of 2, 
otherwise, some goals would be allocated twice as many resources than the 
others. In the above example, where the \conde{} has five clauses, the clauses 
of \texttt{b}, \texttt{c}, and \texttt{d} are allocated more resources.

We end this subsection with precise definitions of all levels of 
\disj{} fairness. Our definition of \emph{fair} \disj{} is slightly generalize 
from the one by \citet{seres1999algebra}. Their definition is for binary 
disjunction. We generalize it to multi-arity one.

\begin{defn}[fair \disj{}]
A \disj{} is fair if and only if it allocates computational resource evenly to 
search spaces produced by goals in the same disjunction (i.e. clauses in 
the same \conde).
\end{defn}

\begin{defn}[almost-fair \disj{}]
A \disj{} is almost-fair if and only if it allocates computational resource so 
evenly to 
search spaces produced by goals in the same disjunction that the maximal ratio 
of resources is bounded by a constant.
\end{defn}

\begin{defn}[unfair \disj{}]
A \disj{} is unfair if and only if it is not even almost-fair.
\end{defn}

\subsection{fair \texttt{conj}}

\NOTE[LKC]{Checked grammer of text before this line}

Given the following program, it is natural to expect lists of each letter to
constitute $1/4$ in the answer. Search strategies with unfair \conj{} (e.g. 
\DFSi, \DFSbi, \DFSf), however, results in many more lists of \texttt{a}s than 
lists of other letters. And some letters are rarely seen. The situation would 
be exacerbated if \conde contains more clauses.

\NOTE[LKC]{Should I add a footnote?

``
Although 
DFS$_\textrm{i}$'s \disj{} is unfair in general, it is fair when there is no 
call to relational definition in \conde{} clauses (including this case).
''}

\begin{center}
	\begin{tabular}{c}
		\lstinputlisting{Figures/repeato-conj-DFSi.rkt}
	\end{tabular}
\end{center}

\begin{center}
	\begin{tabular}{c}
		\lstinputlisting{Figures/repeato-conj-DFSf.rkt}
	\end{tabular}
\end{center}

\begin{center}
	\begin{tabular}{c}
		\lstinputlisting{Figures/repeato-conj-DFSbi.rkt}
	\end{tabular}
\end{center}

Under the hood, the \conde{} and the call to \texttt{repeato} are connected by 
\conj{}. The \conde{} goal outputs a search space including four trivially 
different states. Then the next conjunctive goal, \texttt{(repeato x q)}, is 
applied to each of these states, producing four trivially different search 
spaces. In the examples above, the \conj{}s are allocating more computational 
resources to the search space of \texttt{a}. On the contrary, fair \conj{} 
would allocate resources evenly to each search space. 

\begin{center}
	\begin{tabular}{c}
		\lstinputlisting{Figures/repeato-conj-fair.rkt}
	\end{tabular}
\end{center}

A more interesting situation is when the first conjunct produces infinite many 
answers. Consider the following example, a naive specification of fair \conj{} 
might require search strategies to produce all sorts of singleton lists, but no 
longer ones, which makes the strategies incomplete. A search strategy is 
\emph{complete} if and only if it can find out all the answers within finite 
time, otherwise, it is \emph{incomplete}.

\begin{center}
	\begin{tabular}{c}
		\lstinputlisting{Figures/repeato-conj-infinite-naive.rkt}
	\end{tabular}
\end{center}

Our solution requires a search strategy with \emph{fair} \conj{} to organize
states in bags in search spaces, where each bag contains finite states, and 
to allocate resources evenly among search spaces derived from states in the 
same bag. It is up to a search strategy designer to decide by what criteria to 
put states in the same bag, and how to allocate resources among search spaces 
related to different bags.

BFS puts states of the same cost in the same bag, and allocate resources 
carefully among search spaces related to different bags such that answers are 
produced in increasing order of cost. The \emph{cost} of a answer is its depth 
in the search tree (i.e. the number of calls to relational definitions required 
to find them) \citep{seres1999algebra}. In the following example, the cost of 
each answer is equal to the length of the inner lists plus the length of the 
outer list. 

\begin{center}
	\begin{tabular}{c}
		\lstinputlisting{Figures/repeato-conj-infinite-sof.rkt}
	\end{tabular}
\end{center}

We end this subsection with precise definitions of all levels of \conj{} 
fairness. 

% Our definition of fair \conj{} is orthogonal with completeness. For 
% example, a naively fair strategy is fair but not complete, while BFS is fair and 
% complete.

\begin{defn}[fair \conj{}]
A \conj{} is fair if and only if it allocates computational resource evenly to 
search spaces produced from states in the same bag. A bag is a finite 
collection of state. And search strategies with fair \conj{} should represent 
search spaces with possibly infinite collection of state. 
\end{defn}

\begin{defn}[unfair \conj{}]
A \conj{} is unfair if and only if it is not fair.
\end{defn}





\section{interleaving depth-first search}

In this section, we review the implementation of interleaving depth-first 
search 
(\DFSi). This review is for comparison with other search strategies in this 
paper. Thus we focus on parts that are changed later. TRS2 
\citep{friedman_reasoned_2018} provides a comprehensive description of the 
whole miniKanren implementation.

\begin{figure}
    \lstinputlisting{Figures/DFSi.rkt}
    \caption{implementation of \DFSi}
    \label{DFSi}
\end{figure}

Fig.~\ref{DFSi} shows part of the implementation of \DFSi{}. We follow a
convention to name variables bound to states with `s', to name variables bound 
to goals with `g', and to name variables bound to search spaces with a suffix 
`-inf'. The first function, \texttt{disj2}, implements binary disjunction. 
\texttt{append-inf} is its helper, which composes two disjunctive search 
spaces. 
The following function, \texttt{conj2}, implements binary conjunction. It 
applies 
the \emph{first} goal to the input state, then applies the second goal to 
states in the resulting search space. The latter process is done with a helper 
function. \texttt{append-map-inf} applies its input goal to states in its input 
search spaces and compose the resulting search spaces. It reuses 
\texttt{append-inf} for search space composition. The following three 
definitions introduce syntactic sugars that miniKanren users are more familiar 
with. The first two definitions say disjunction and disjunction are 
right-associative. The next and last definition says \conde{} relates its 
clauses disjunctively, and goals in the same clause conjunctively.

\section{balanced interleaving depth-first search}

\begin{figure}
	\lstinputlisting{Figures/balanced-disj.rkt}
	\caption{implementation of DFS$_\textrm{bi}$}
	\label{balanced-disj}
\end{figure}

Balanced interleaving DFS (DFS$_\textrm{bi}$) has almost-fair \disj{} and unfair 
\conj{}. The implementation of DFS$_\textrm{bi}$ differs from 
DFS$_\textrm{i}$'s 
in the \disj{} macro. We list the new \disj{} with its helper in 
Fig.~\ref{balanced-disj}. The helper \texttt{disj+} builds a balanced binary 
tree 
whose leaves are the goals and whose nodes are \texttt{disj2}s, hence the name 
of 
this search strategy. The first argument 
to \texttt{disj+} is a list of all goals. And the next two arguments accumulate 
goals in 
the left and right subtrees. The first clause says that when there is one goal, 
the tree is the goal itself. When there are more goals, the first argument is 
partitioned into two sub-lists. The partition is done by repetitively 
dispatching the first two goals (the last clause), until no goal remains (the 
second clause) or one goal remains (the third clause). In contrast, the \disj{} 
in DFS$_\textrm{i}$ constructs the binary tree with the same collection of leaf 
nodes but in a particularly unbalanced form.


% The first helper function, \texttt{split}, takes a 
% list of goals \texttt{ls} and a procedure \texttt{k}, partitions \texttt{ls} 
% into two sub-lists of roughly equal length, and returns the application of 
% \texttt{k} to the two sub-lists. \texttt{disj*} takes a non-empty list of goals 
% \texttt{gs} and returns a goal. With the help of \texttt{split}, it essentially 
% constructs a \emph{balanced} binary tree where leaves are elements of 
% \texttt{gs} and nodes are \texttt{disj2}s, hence the name of this search 
% strategy. 

\section{fair depth-first search}

\begin{figure}
	\lstinputlisting{Figures/fDFS.rkt}
	\caption{implementation of DFS$_\textrm{f}$}
	\label{fDFS}
\end{figure}

Fair DFS (\DFSf) has fair \disj{} and unfair \conj{}. The 
implementation of \DFSf{} differs from \DFSi{}'s in 
\texttt{disj2} (Fig.~\ref{fDFS}). \texttt{disj2} is changed to call a new and 
fair version of \texttt{append-inf}. \texttt{append-inf/fair} immediately calls 
its helper, \texttt{loop}, with the first argument, \texttt{s?}, set to 
\texttt{\#{}t}, which indicates that 
\texttt{s-inf} and \texttt{t-inf} haven't been swapped. The swapping happens at 
the third \texttt{cond} clause in the helper, where \texttt{s?} is updated 
accordingly. The first two \texttt{cond} clauses essentially copy the 
\texttt{car}s and stop recursion when one of the input spaces is obviously 
finite. The third clause, as we mentioned above, is just for swapping. When the 
fourth and last clause runs, we know that both \texttt{s-inf} and \texttt{t-inf} 
are ended with a thunk. In this case, a new thunk is constructed. The new thunk 
calls the driver recursively. Here changing the order of \texttt{t-inf} and 
\texttt{s-inf} won't hurt the fairness (though it will change the order of 
answers). We swap them back so that answers are produced in a more natural 
order.


\section{breadth-first search}

BFS has both fair \disj{} and fair \conj{}. Our implementation is based on 
DFS$_\textrm{f}$ (not DFS$_\textrm{i}$). To implement \BFS{} based on \DFSf{}, 
we need \texttt{append-map-inf/fair} in addition to \texttt{append-inf/fair}. 
The only difference between \texttt{append-map-inf/fair} and 
\texttt{append-map-inf} is that the former calls \texttt{append-inf/fair} 
instead of \texttt{append-inf}.

The implementation can be improved in two ways. First, as mentioned in 
section 2.2, BFS puts answers in bags and answers of the same cost are in the 
same bag. In this implementation, however, it is unclear where this information 
is recorded. Second, \texttt{append-inf/fair} is extravagant in memory usage. It 
makes $O(n+m)$ new \texttt{cons} cells every time, where $n$ and $m$ are the 
``length''s of input search spaces. We address these issues in the first 
subsection.

Both our BFS and Seres's BFS \citet{seres1999algebra} produce answers in 
increasing order of cost. So it is interesting to see if they are equivalent. 
We prove so in Coq. The details are in the second subsection.

\subsection{optimized BFS}

\begin{figure}
	\lstinputlisting{Figures/BFS-opt.rkt}	
	\caption{new and changed functions in optimized BFS that implements pure 
	features}
	\label{BFS-opt}
\end{figure}

As mentioned in section 2.2, BFS puts answers in bags and answers of the 
same cost are in the same bag. The cost
information is recorded subtly -- the \texttt{car}s of a search space have cost 
0 (i.e. they are in the same bag), and the costs of answers in thunk are 
computed recursively then increased by one. It is even more subtle that
\texttt{append-inf/fair} and the \texttt{append-map-inf/fair} respects the cost 
information. We make these facts more obvious by changing the type of search 
space, modifying related function definitions, and introducing a few more 
functions.

The new type is a pair whose \texttt{car} is a list of answers (the bag), and 
whose \texttt{cdr} is either a \texttt{\#{}f} or a thunk returning a search 
space. A falsy \texttt{cdr} means the search space is obviously finite. 

Functions related to the pure subset are listed in Fig.~\ref{BFS-opt} (the 
others in Fig.~\ref{BFS-opt-cont}). They are compared with 
\citeauthor{seres1999algebra}'s implementation later. The first three functions 
in Fig.~\ref{BFS-opt} are search space constructors. \texttt{none} makes an 
empty search space; \texttt{unit} makes a space from one answer; and 
\texttt{step} makes a space from a thunk. The remaining functions do the same 
thing as before. 

Luckily, the change in \texttt{append-inf/fair} also fixes the miserable space 
extravagance -- the use of \texttt{append} helps us to reuse the first bag of 
\texttt{t-inf}.

\citet{kiselyov2005backtracking} has shown that a \emph{MonadPlus} hides in 
implementations of logic programming system. Our BFS implementation is not an 
exception: \texttt{append-map-inf/fair} is like \texttt{bind}, 
but takes arguments in reversed order; \texttt{none}, \texttt{unit}, and 
\texttt{append-inf/fair} correspond to \texttt{mzero}, \texttt{unit}, 
and \texttt{mplus} respectively.

\begin{figure}
	\lstinputlisting{Figures/BFS-opt-cont.rkt}	
	\caption{new and changed functions in optimized BFS that implements impure 
		features}
	\label{BFS-opt-cont}
\end{figure}

Functions implementing impure features are in Fig.~\ref{BFS-opt-cont}. The 
first function, \texttt{elim}, takes a space \texttt{s-inf} and two 
continuations \texttt{ks} and \texttt{kf}. When \texttt{s-inf} contains some 
answers, \texttt{ks} is called with the first answer and the rest space. 
Otherwise, \texttt{kf} is called with no argument. Here `s' and `f' means 
`succeed' and `fail' respectively. This function is like an eliminator of 
search space, hence the name. The remaining functions do the same thing as 
before.

\subsection{comparison with the BFS of \citet{seres1999algebra}}

In this section, we compare the pure subset of our optimized BFS with the BFS 
found in \citet{seres1999algebra}. We focus on the pure subset because 
Silvija's system is pure. Their system represents search spaces with streams of 
lists of answers, where each list is a bag.

To compare efficiency, we translate her Haskell code into Racket (See 
supplements for the translated code). The translation is direct 
due to the similarity in both logic programming systems and search space 
representations. The translated code is longer and slower than our BFS 
implementation. Details about 
difference in efficiency are in section 6.

We prove in Coq that the two BFSs are equivalent, i.e. \texttt{(run n g)} 
produces the same result (See supplements for the formal proof).

\section{quantitative evaluation}

\begin{table}
	\begin{tabular}{|c|c|c|c|c|c|c|}
		\hline 
		benchmark & size & DFS$_\textrm{i}$ & DFS$_\textrm{bi}$ & DFS$_\textrm{f}$ & optimized BFS & Silvija's BFS  
		\\
		\hline
		very-recursiveo & 100000 &  579 &  793 & 2131 & 1438 & 3617 \\
		& 200000 & 1283 & 1610 & 3602 & 2803 & 4212 \\
		& 300000 & 2160 & 2836 &    - & 6137 &    - \\
		\hline 
		appendo  & 100 &  31 &  41 &  42 &  31 &  68 \\ 
		& 200 & 224 & 222 & 221 & 226 & 218 \\ 
		& 300 & 617 & 634 & 593 & 631 & 622 \\ 
		\hline 
		reverso & 10 &   5 &   3 &   3 &     38 &     85 \\ 
		& 20 & 107 &  98 &  51 &   4862 &   5844 \\
		& 30 & 446 & 442 & 485 & 123288 & 132159 \\ 
		\hline
		quine-1 & 1 &  71 &  44 & 69 & - & - \\ 
		& 2 & 127 & 142 & 95 & - & - \\ 
		& 3 & 114 & 114 & 93 & - & - \\ 
		\hline
		quine-2 & 1 & 147 & 112 &  56 & - & - \\ 
		& 2 & 161 & 123 & 101 & - & - \\ 
		& 3 & 289 & 189 & 104 & - & - \\ 
		\hline 
		'(I love you)-1 &  99 & 56 & 15 & 22 &  74 & 165 \\ 
		& 198 & 53 & 72 & 55 &  47 &  74 \\
		& 297 & 72 & 90 & 44 & 181 & 365 \\ 
		\hline
		'(I love you)-2 &  99 & 242 &  61 & 16 &  66 &  99 \\ 
		& 198 & 445 & 110 & 60 &  42 &  64 \\
		& 297 & 476 & 146 & 49 & 186 & 322 \\ 
		\hline 
	\end{tabular}
	\caption{The results of a quantitative evaluation: running times of 
	benchmarks 
		in milliseconds}
	\label{compare-efficiency}
\end{table}

In this section, we compare the efficiency of search strategies. A concise 
description is in Table~\ref{compare-efficiency}. A hyphen means running out of 
memory. The first two benchmarks are taken from 
\citet{friedman_reasoned_2018}. \texttt{reverso} is from 
\citet{rozplokhas2018improving}. Next two benchmarks 
about quine are modified from a similar test case in \citet{byrd2017unified}. 
The modifications are made 
to circumvent the need for symbolic constraints (e.g. $\neq$, 
\texttt{absento}). Our version generates de 
Bruijnized expressions and prevent closures getting into list. The two 
benchmarks differ in the \conde clause order of their relational interpreters. 
The last two 
benchmarks are about synthesizing expressions that evaluate to \texttt{'(I love 
you)}. This benchmark is also inspired by \citet{byrd2017unified}. Again, the 
sibling benchmarks differ in the \conde clause order of their relational 
interpreters. The first one 
has elimination rules (i.e. application, \texttt{car}, and \texttt{cdr}) at the 
end, while the other has them at the beginning. We conjecture that DFS$_\textrm{i}$ would 
perform badly in the second case because elimination rules complicate the 
problem when running backward. The evaluation supports our conjecture.

In general, only DFS$_\textrm{i}$ and DFS$_\textrm{bi}$ constantly perform well. DFS$_\textrm{f}$ is just as 
efficient in all benchmarks but \texttt{very-recursiveo}. Both BFS have obvious 
overhead in many cases. Among the three variants of DFS (they all have unfair 
\conj{}), DFS$_\textrm{f}$ is most resistant to clause permutation, followd by DFS$_\textrm{bi}$ then 
DFS$_\textrm{i}$. Among the two implementation of BFS, ours constantly performs as well or 
better. Interestingly, every strategies with fair \disj{} suffers in 
\texttt{very-recursiveo} and DFS$_\textrm{f}$ performs well elsewhere. 
Therefore, this 
benchmark might be a special case. Fair \conj{} imposes overhead constantly 
except in \texttt{appendo}. The reason might be that strategies with fair 
\conj{} tend to keep more intermediate answers in the memory.

\section{related works}

Edward points out a disjunct complex would be `fair' if it is a full and 
balanced tree \citet{yang2010adventures}.

Silvija et al \citet{seres1999algebra} also describe a breadth-first search 
strategy. We proof their BFS is equivalent to ours. But our code looks simpler 
and performs better in comparison with a straightforward translation of their 
Haskell code.

\section{conclusion}

We analysis the definitions of fair \disj{} and fair \conj{}, then propose a 
new definition of fair \conj{}. Our definition is orthogonal with completeness.

We devise two new search strategies (i.e. balanced interleaving DFS 
(DFS$_\textrm{bi}$) and fair DFS (DFS$_\textrm{f}$)) and devise a new 
implementation of BFS. These strategies have different features 
in fairness: $_\textrm{bi}$ has almost-fair \disj{} and unfair \conj{}. 
DFS$_\textrm{f}$ has fair \disj{} and unfair \conj{}. BFS has both fair \disj{} 
and fair \conj{}.

Our quantitative evaluation shows that DFS$_\textrm{bi}$ and DFS$_\textrm{f}$ are competitive 
alternatives to DFS$_\textrm{i}$, the current search strategy, and that BFS is less 
practical.

We prove our BFS is equivalent to the BFS in \citet{seres1999algebra}. Our code 
is shorter and runs faster than a direct translation of their Haskell code.

\section*{acknowledgments}

\bibliographystyle{ACM-Reference-Format}
\bibliography{citation}

\end{document}

